

\begin{frame}
\thetitle{A Language Model}
    Our goal is to model a sentence, $x_1 \ldots x_T$.
    
    \air 
    Context: RNN language models are remarkable at this task,
    \[ x_{1: T}\sim  \RNNLM(x_{1:T}; \theta). \]
     
    Defined as, 
    \begin{eqnarray*}
     p(x_{1:T})  = \displaystyle \prod_{t=1}^T p(x_t \given x_{<t}) =  \displaystyle \prod_{t=1}^T \softmax(\boldW \RNN(\boldh_t, \boldx_t;\theta))
    \end{eqnarray*}
    % \[ p(x_{1:T})  = \prod_{t=1}^T p(x_t \given x_{<t}) = \prod_{t=1}^T \softmax(\RNN(\boldh_{t-1}, x_t;\theta)). \]
\begin{center}
\begin{tikzpicture}
% nodes
\node (dots) {$\ldots$};%
 \node[obs, left=1cm of dots] (x1) {$x_1^{(n)}$};%
 \node[obs, right=1cm of dots] (xT) {$x_T^{(n)}$};%
 \node[const, below=1cm of dots] (pi) {$\theta$};
 
% plate
 \plate {plate1} {(dots)(x1)(xT)} {$N$}; %
% edges
 \edge {x1} {dots};
 %\edge[bend left] {x1.south} {xT.south};
  \edge {dots} {xT};
 \edge {pi} {x1,xT.south};

 \draw[->] 
 (x1) edge[bend right] node [right] {} (xT);
 %\draw[->]
 %(dots) edge[bend right] node [right] {} (xT);
 
\end{tikzpicture}
\end{center}
\end{frame}

%\begin{frame}
%\thetitle{What if your goal is not just model fit?}

%Other aims:
%\begin{itemize}
%    \item Small data, low parameter
%    \item Interpretable models
%    \item Structure estimation
%\end{itemize}
%\pause
%This section: Three deep latent variable alternatives
%\begin{itemize}
%    \item Discrete Variables
%    \item Continuous Variables
%    \item Structured Variables
%\end{itemize}
%
%\end{frame}

\begin{frame}
    \thetitle{A Collection of Model Archetypes}
    Focus: semi-supervised or unsupervised learning, i.e. don't just learn the probabilities, but the process.
    \vspace{0.5cm}
    
    Range of choices in selecting $z$ 
    \begin{enumerate}
        \item Discrete LVs $z$ (``Clustering'')
        \item Continuous LVs $z$ (``Dimensionality reduction'')
        \item Structured LVs $z$ (``Structured prediction'')
    \end{enumerate}
    We consider one classical / one deep model for each.
    
    
    \vspace{0.5cm}
    Main goal will be posterior inference / Bayes rule.
    \[ p(z \param x ) \propto p(z; \theta) p(x \given z ;\theta) \]
\end{frame}

\subsection{Discrete Models}
\begin{frame}
\thetitle{Model 1: Discrete - Clustering}

\begin{center}
\begin{tikzpicture}
% nodes
\node[draw, text width=5cm, rounded corners] (x1) {\footnotesize In an old house in Paris that was covered with vines lived twelve little girls in two straight lines.};%

\node[draw, right=of x1, rounded corners] (x3) {Cluster 23};%
\draw[->] (x1) -- (x3);
\end{tikzpicture}
\end{center}



Categorical latent variable models induce a clustering over sentences $x^{(n)}$

 \begin{itemize}
     \item Document/sentence clustering~\citep{willett1988recent,aggarwal2012survey}.
     \item Mixture of expert text generation models~\citep{jacobs1991adaptive,garmash2016ensemble,lee2016stochastic}
 \end{itemize}

\end{frame}

\begin{frame}
\thetitle{Model 1: Discrete - Mixture of Multinomials}

\vspace{0.25cm}

Generative process: 
\begin{enumerate}
\item Draw cluster $z \in \{1, \ldots, K\}$ from a Categorical with param $\mu$.
\item For each $t$, draw word $x_t$ from Categorical with word distribution $\pi_{z}$.
\end{enumerate}
\air
Parameters: $\theta = \{\mu, \pi\}$

\air



Gives rise to the "Naive Bayes" distribution:
\begin{align*}
     p(x, z \param \theta) = p(z \param \mu) \times p(x \given z \param \pi) = \mu_{z} \times  \prod_{t=1}^T \pi_{z, x_t}
\end{align*}
\end{frame}


\begin{frame}
\thetitle{Model 1: Graphical Model View}

%\scalebox{0.75}{
\begin{center}

\begin{tikzpicture}
  %\tikz{
% nodes
\node (dots) {$\ldots$};%
 \node[obs, left=1cm of dots] (x1) {$x_1^{(n)}$};%
 \node[obs, right=1cm of dots] (xT) {$x_T^{(n)}$};%
 \node[latent, above=of dots] (z) {$z^{(n)}$}; %
 \node[const, above=(0.5 cm) of z] (mu) {$\mu$};
 \node[const, below left=0.3cm and 0.8cm of x1] (pi) {$\pi$};
 
% plate
 \plate {plate1} {(dots)(x1)(xT)(z)} {$N$}; %
% edges
 \edge {z} {dots};
 \edge {z} {x1};
 \edge {z} {xT};
 \edge {mu} {z};
 \edge {pi.east} {x1,xT};
 %}
 \end{tikzpicture}
 %}
\end{center}

\air
\air

\begin{align*}
     \prod_{n=1}^N p(x^{(n)}, z^{(n)} \param \mu, \pi) &= \prod_{n=1}^N p(z^{(n)} \param \mu) \times p(x^{(n)} \given z^{(n)} \param \pi) \\
     &= \prod_{n=1}^N \mu_{z^{(n)}}   \prod_{t=1}^T \pi_{z^{(n)},  x^{(n)}_t}
\end{align*}
\end{frame}





\begin{frame}
\thetitle{Deep Model 1: Discrete - Mixture of RNNs}

Generative process: 
\begin{enumerate}
\item Draw cluster $z \in \{1, \ldots, K\}$ from a Categorical.
\item Draw words $x_{1:T}$ from RNNLM with parameters $\pi_z$.
\end{enumerate}
\[p(x, z \param \theta) 
       = \mu_{z} \times   \RNNLM(x_{1:T} \param \pi_z) \]
% \[p(x, z \param \theta) 
%       = \mu_{z} \times  \prod_{t=1}^T \softmax(\RNN(\boldh_{t-1}, x_t\param \pi_z))\]
\begin{center}

\begin{tikzpicture}
  %\tikz{
% nodes
\node (dots) {$\ldots$};%
 \node[obs, left=1cm of dots] (x1) {$x_1^{(n)}$};%
 \node[obs, right=1cm of dots] (xT) {$x_T^{(n)}$};%
 \node[latent, above=of dots] (z) {$z^{(n)}$}; %
 \node[const, above=(0.5cm) of z] (mu) {$\mu$};
 \node[const, below left=0.3cm and 0.8cm of x1] (pi) {$\pi$};
 
% plate
 \plate {plate1} {(dots)(x1)(xT)(z)} {$N$}; %
% edges
 \edge {z} {dots};
 \edge {z} {x1};
 \edge {z} {xT};
 \edge {mu} {z};
 \edge {pi.east} {x1,xT.south};
 \edge {x1} {dots};
 %\edge[bend left] {x1.south} {xT.south};
  \edge {dots} {xT};

 \draw[->] 
 (x1) edge[bend right=10] node [right] {} (xT);
 %}
 \end{tikzpicture}
 %}
\end{center}
%\begin{align*}
%\boldh_{z,t} &= \tanh(\mathbf{W}_z \bolde_t +\mathbf{U}_z\boldh_{z,t-1}  + \boldb_{z}) \nonumber \\
%p(x_{t} \given x_{<t} , z) &= \softmax(\mathbf{V} \boldh_{z,t-1} + \boldc)_{x_{t}} \nonumber \\
%p(x_1, \ldots, x_T \given z) &= \prod_{t=1}^{T} p(x_{t} \given x_{<t} , z) 
%\end{align*}

\pause
\air
\air
\air

\end{frame}


\begin{frame}
\thetitle{Not so Naive, Still a bit Bayes}
Note the difference in the number of token parameters: 
\begin{itemize}
    \item Order $K \times V $ Categorical parameters.
    \item Order $K \times d^2 + V \times d$ with RNN with $d$ hidden dims
\end{itemize}

\vspace{0.5cm}

Bayes Rule still possible, but requires more work,
\begin{itemize}
    \item Posterior requires running all $K$ RNNs: \[p(z \given x) \propto \mu_{z} \times \displaystyle \RNNLM(x_{1:T} \param \pi_z) \]
% \[p(z \given x) \propto \mu_{z} \times \displaystyle \prod_{t=1}^T \softmax(\RNN(\boldh_{t-1}, x_t\param \pi_z))\]    
\end{itemize}

    
\end{frame}

% \begin{frame}
% \thetitle{Making the Model ``Deep'' II}
% Rather than generate tokens, generate \textit{embeddings} of tokens:


% Parameterize the token distribution with neural components \textit{and} generate embeddings, rather than discrete objects:


% \pause
% \air
% \air
% \air

% Note the difference in the number of token parameters: 
% \begin{itemize}
%     \item $K \times V \times T$ under Naive Bayes 
%     \item $K \times d(2d+1) + V \times (d+1)$ with RNN if $\boldh_t, \bolde_t \in \reals^d$
% \end{itemize}
% \end{frame}

\subsection{Continuous Models}

\begin{frame}
\thetitle{Model 2: Continuous - Dimensionality Reduction}

\begin{center}
\begin{tikzpicture}
% nodes
\node[draw, text width=5cm, rounded corners] (x1) {\footnotesize In an old house in Paris that was covered with vines lived twelve little girls in two straight lines.};%

\begin{scope}[xshift=5cm]
\draw (-1, 0) -- (1, 0);
\draw (0, -1) -- (0, 1);
\draw (0.5, 0.5) circle (0.05);
\end{scope}

\node[draw=white, right=of x1, rounded corners] (x3) {};%

\draw[->] (x1) -- (x3);

\end{tikzpicture}
\end{center}

Find a lower-dimensional, well-behaved  continuous representation of a sentence.
\air 

Latent variables in $\reals^d$ make distance/similarity easy. 

\begin{itemize}
    \item Certain sentence embeddings (e.g., Skip-Thought vectors~\citep{kiros2015skip}) can be interpreted in this way.
\end{itemize}

\pause
Continuous-value more convenient from a VAE perspective:
\begin{itemize}
    \item Recent work in text generation assumes a latent vector per sentence~\citep{Bowman2016,Yang2017,hu2017toward}.
\end{itemize}
\end{frame}

\begin{frame}
\thetitle{Model 2: Continuous - Sentence Topic Model}
\vspace{0.25cm}

Generative Process:
\begin{enumerate}
    \item Draw continuous latent variable $\boldz$ from Dirichlet with param $\alpha$.
    \item For each $t$, draw word $x_t$ from Categorical with param $z$.
\end{enumerate}
\air

Parameters: $\theta= \{\alpha\}$

\air

Intuition: $\alpha$ is the distribution over words generally, $\boldz$ captures the topical word distribution of the sentence. 
\end{frame}

\begin{frame}
\thetitle{Graphical Model View}

%\scalebox{0.75}{
\begin{center}

\begin{tikzpicture}
  %\tikz{
% nodes
\node (dots) {$\ldots$};%
 \node[obs, left=1cm of dots] (x1) {$x_1^{(n)}$};%
 \node[obs, right=1cm of dots] (xT) {$x_T^{(n)}$};%
 \node[latent, above=of dots] (z) {$z^{(n)}$}; %
 \node[const, above=of z] (mu) {$\alpha$};
 %\node[const, below left=0.3cm and 0.8cm of x1] (pi) {$\theta$};
% plate
 \plate {plate1} {(dots)(x1)(xT)(z)} {$N$}; %
% edges
 \edge {z} {dots};
 \edge {z} {x1};
 \edge {z} {xT};
 \edge {mu} {z};
 %\edge {pi.east} {x1,xT};
 %}
 \end{tikzpicture}
 %}
\end{center}

\air
\air

Gives rise to the joint distribution:
\begin{align*}
     \prod_{n=1}^N p(x^{(n)}, z^{(n)} \param \alpha) &= \prod_{n=1}^N p(z^{(n)} \param \alpha) \times p(x^{(n)} \given z^{(n)} ) \\
     &= \prod_{n=1}^N \prod_{v=1}^V  (z^{(n)})^{\alpha_v + \sum_{t=1}^T  1(x^{(n)}_t = v)}  \\
     \end{align*}
\end{frame}


\begin{frame}
\thetitle{Deep Model 2: Continuous - Topic Latent}
Generative Process:

\begin{enumerate}
    \item Draw latent variable $\boldz \sim \mcN(\bmu, \bSigma)$.
    \item Draw each token $x_t$ from a conditional RNNLM. 
\end{enumerate}


RNN is also conditioned on latent $\boldz$,  
\begin{align*}
     & p(x, \boldz \param \pi, \bmu, \bSigma)  = p(\boldz \param \bmu, \bSigma) \times p(x \given \boldz \param \pi) \\ 
      &  = \mcN(\boldz \param \bmu, \bSigma)  \times \RNNLM(\boldx_{1:T}  \param \pi,  \boldz)
    %   &  = \mcN(z\param \mu, \Sigma)  \prod_{t=1}^T \softmax(\RNN(\boldh_{t-1}, \MLP([x_t, z]); \pi))      
\end{align*}

where 
\[ \RNNLM(\boldx_{1:T}  \param \pi,  \boldz) = \prod_{t=1}^T \softmax(\boldW \RNN(\boldh_{t-1}, [x_t, z]; \pi)) \]

\end{frame}

\begin{frame}
\thetitle{Graphical Model View}

\begin{center}
\begin{tikzpicture}
% nodes
\node (dots) {$\ldots$};%
 \node[obs, left=1cm of dots] (x1) {$x_1^{(n)}$};%
 \node[obs, right=1cm of dots] (xT) {$x_T^{(n)}$};%
 \node[latent, above=of dots] (z) {$\boldz^{(n)}$}; %
 \node[const, above left=1cm and 0.25cm of z] (mu) {$\bmu$};
 \node[const, above right=1cm and 0.25cm of z] (sigma) {$\bSigma$};
 \node[const, below left=0.6cm and 0.6cm of x1] (pi) {$\pi$};
 
% plate
 \plate {plate1} {(dots)(x1)(xT)(z)} {$N$}; %
% edges
 \edge {z} {dots};
 \edge {z} {x1};
 \edge {z} {xT};
 \edge {mu} {z};
 \edge {sigma} {z};
 \edge {x1} {dots};
 %\edge[bend left] {x1.south} {xT.south};
  \edge {dots} {xT};
 \edge {pi.east} {x1,xT.south};
 
 \draw[->] 
 (x1) edge[bend right] node [right] {} (xT);
 %\draw[->]
 %(dots) edge[bend right] node [right] {} (xT);
 
\end{tikzpicture}
\end{center}
    
\end{frame}



\subsection{Structured Models}

\begin{frame}
\thetitle{Model 3: Structured Prediction}
\begin{center}
\begin{tikzpicture}
% nodes
\node[draw, text width=5cm, rounded corners] (x1) {\footnotesize In an old house in Paris that was covered with vines lived twelve little girls in two straight lines.};%

\begin{scope}[xshift=5cm]
\draw (0, 1) -- (0.5, 0.5);
\draw (0, 1) -- (-0.5, 0.5);
\draw (-0.75, 0) -- (-0.5, 0.5);
\draw (-0.25, 0) -- (-0.5, 0.5);

%\draw () -- (-0.5, 0.5);

\end{scope}

\node[draw=white, right=of x1, rounded corners] (x3) {};%

\draw[->] (x1) -- (x3);

\end{tikzpicture}
\end{center}


Structured latent variable models are often used when we would like to infer unannotated structure:

\begin{itemize}
    \item Unsupervised POS tagging~\citep{brown1992class,merialdo1994tagging,smith2005contrastive}
    \item Unsupervised dependency parsing~\citep{klein2004corpus,headden2009improving}
\end{itemize}


Or when structure is useful for \textit{interpreting} our data:
\begin{itemize}
    \item Segmentation of documents into topical passages~\citep{hearst1997texttiling}
    \item Alignment~\citep{vogel1996hmm}
\end{itemize}
\end{frame}

\begin{frame}
\thetitle{Model 3: Structured - Hidden Markov Model}
Generative Process:

\begin{enumerate}
    \item For each $t$, draw $z_t \in \{1, \ldots, K\}$ from a Categorical with param $\mu_{z_{t-1}}$.
    \item Draw observed token $x_t$ from Categorical with param $\pi_{z_{t}}$.
\end{enumerate}
\air
Parameters: $\theta = \{ \mu, \pi\}$
\air

\pause
Gives rise to the joint distribution:
\begin{alignat*}{2}
    p(x, z; \ \theta)  &= \prod_{t=1}^T p(z_t \given z_{t-1} ; \ \mu_{z_{t-1}}) &&\times \prod_{t=1}^T p(x_t \given z_{t} ; \ \pi_{z_{t}}) \nonumber \\ 
    &= \prod_{t=1}^T \mu_{z_{t-1}, z_t} &&\times \prod_{t=1}^T \pi_{z_{t}, x_t}
\end{alignat*}

\end{frame}


\begin{frame}
\thetitle{Graphical Model View}
\begin{center}
%\scalebox{0.8}{
\begin{tikzpicture}
  %\tikz{
% nodes
 \node[obs] (x1) {$x_1$};
 
 \node[obs, right=1cm of x1] (x2) {$x_2$};%
 \node[obs, right=1cm of x2] (x3) {$x_3$};%
 \node[obs, right=1cm of x3] (x4) {$x_4$};%
 \node[latent, above=of x1] (z1) {$z_1$}; %
 \node[latent, above=of x2] (z2) {$z_2$}; %
 \node[latent, above=of x3] (z3) {$z_3$}; %
 \node[latent, above=of x4] (z4) {$z_4$}; %
  \node[const, above left=0.5cm and 0.25cm of z3] (mu) {$\mu$};
  \node[const, below left=0.75cm and 0.25cm of x3] (theta) {$\pi$};

 %\node[const, above left=1cm and 0.5cm of z3] (A) {$\boldA$};
 %\node[const, below left=1cm and 0.5cm of x3] (B) {$\boldB$};
 
% edges

 \edge {mu} {z1,z2,z3,z4};
 \edge {z1} {x1,z2};
 \edge {z2} {x2,z3};
 \edge {z3} {x3,z4};
 \edge {z4} {x4};
 \edge {theta} {x1,x2,x3,x4};

 %\edge {A}{z1,z2,z3,z4};
 %\edge {B}{x1,x2,x3,x4};
 %}
 \plate {plate1} {(x1)(x2)(x3)(x4)(z1)(z2)(z3)(z4)} {$N$}; %
 \end{tikzpicture}
 %}
\end{center} 

\begin{alignat*}{2}
    p(x, z; \ \theta)  &= \prod_{t=1}^T p(z_t \given z_{t-1} ; \ \mu_{z_{t-1}}) &&\times \prod_{t=1}^T p(x_t \given z_{t} ; \ \pi_{z_{t}}) \nonumber \\ 
    &= \prod_{t=1}^T \mu_{z_{t-1}, z_t} &&\times \prod_{t=1}^T \pi_{z_{t}, x_t}
\end{alignat*}

\end{frame}

\begin{frame}
\thetitle{Further Extension: Factorial HMM}
A variety of important extended versions. 

\begin{center}
%\scalebox{0.8}{
\begin{tikzpicture}
  %\tikz{
% nodes
 \node[obs] (x1) {$x_1$};
  \node[obs, right=1cm of x1] (x2) {$x_2$};%
 \node[obs, right=1cm of x2] (x3) {$x_3$};%
 \node[obs, right=1cm of x3] (x4) {$x_4$};%

 \node[latent, above=5mm of x1] (z1) {$z_{1,1}$}; %
 \node[latent, above=5mm of x2] (z2) {$z_{1,2}$}; %
 \node[latent, above=5mm of x3] (z3) {$z_{1,3}$}; %
 \node[latent, above=5mm of x4] (z4) {$z_{1,4}$}; %
  \node[latent, above=5mm of z1] (k1) {$z_{2,1}$}; %
 \node[latent, above=5mm of z2] (k2) {$z_{2,2}$}; %
 \node[latent, above=5mm of z3] (k3) {$z_{2,3}$}; %
 \node[latent, above=5mm of z4] (k4) {$z_{2,4}$}; %
   \node[latent, above=5mm of k1] (j1) {$z_{3,1}$}; %
 \node[latent, above=5mm of k2] (j2) {$z_{3,2}$}; %
 \node[latent, above=5mm of k3] (j3) {$z_{3,3}$}; %
 \node[latent, above=5mm of k4] (j4) {$z_{3,4}$}; %
%   \node[const, above left=1cm and 0.25cm of z3] (theta2) {$\theta$};
%   \node[const, below left=1cm and 0.25cm of x3] (theta) {$\theta$};

 %\node[const, above left=1cm and 0.5cm of z3] (A) {$\boldA$};
 %\node[const, below left=1cm and 0.5cm of x3] (B) {$\boldB$};
 
% edges

%  \edge {theta2} {z1,z2,z3,z4};
 \edge {z1} {x1,z2};
 \edge {z2} {x2,z3};
 \edge {z3} {x3,z4};
 \edge {z4} {x4};
  \edge {k1}{k2};
 \edge {k2} {k3};
 \edge {k3} {k4};
  \edge {j1}{j2};
 \edge {j2} {j3};
 \edge {j3} {j4};
 \draw[->] (k4) to[bend left=30] (x4);
 \draw[->] (k3) to[bend left=30] (x3);
 \draw[->] (k2) to[bend left=30] (x2);
 \draw[->] (k1) to[bend left=30] (x1);
  \draw[->] (j4) to[bend left=30] (x4);
 \draw[->] (j3) to[bend left=30] (x3);
 \draw[->] (j2) to[bend left=30] (x2);
 \draw[->] (j1) to[bend left=30] (x1);
%  \edge {theta} {x1,x2,x3,x4};

 %\edge {A}{z1,z2,z3,z4};
 %\edge {B}{x1,x2,x3,x4};
 %}
 \plate {plate1} {(x1)(x2)(x3)(x4)(z1)(z2)(z3)(z4)(k1)(k2)(k3)(k4)(j1)(j2)(j3)(j4)} {$N$}; %
 \end{tikzpicture}
 %}
\end{center} 

\begin{alignat*}{2}
    p(x, z; \ \theta)  &= \prod_{l=1}^L \prod_{t=1}^T p(z_{l,t} \given  z_{l, t-1}) &&\times \prod_{t=1}^T p(x_t \given z_{1:L, t}) \nonumber \\ 
\end{alignat*}

\end{frame}



\begin{frame}
\thetitle{Deep Model 3: Deep HMM}
Parameterize transition and emission distributions with neural networks (c.f., \citet{tran2016})

\air
\begin{itemize}
    \item Model transition distribution as
    \begin{align*}
p(z_t \given z_{t-1}) = \softmax(\MLP(z_{t-1} \param \mu)) 
\end{align*}
\item Model emission distribution as 
\begin{align*}
p(x_t \given z_{t}) = \softmax(\MLP(z_t \param \pi))
\end{align*}    
\end{itemize}

\air
\pause

\textbf{Note:} $K \times K$ transition parameters for standard HMM vs. $O(K \times d + d^2)$ for deep version.    
\end{frame}



\begin{frame}
\thetitle{Graphical Model View}
\begin{center}
%\scalebox{0.8}{
\begin{tikzpicture}
  %\tikz{
% nodes
 \node[obs] (x1) {$x_1$};
 
 \node[obs, right=1cm of x1] (x2) {$x_2$};%
 \node[obs, right=1cm of x2] (x3) {$x_3$};%
 \node[obs, right=1cm of x3] (x4) {$x_4$};%
 \node[latent, above=of x1] (z1) {$z_1$}; %
 \node[latent, above=of x2] (z2) {$z_2$}; %
 \node[latent, above=of x3] (z3) {$z_3$}; %
 \node[latent, above=of x4] (z4) {$z_4$}; %
  \node[const, above left=0.5cm and 0.25cm of z3] (mu) {$\mu$};
  \node[const, below left=0.75cm and 0.25cm of x3] (theta) {$\pi$};

 %\node[const, above left=1cm and 0.5cm of z3] (A) {$\boldA$};
 %\node[const, below left=1cm and 0.5cm of x3] (B) {$\boldB$};
 
% edges

 \edge {mu} {z1,z2,z3,z4};
 \edge {z1} {x1,z2};
 \edge {z2} {x2,z3};
 \edge {z3} {x3,z4};
 \edge {z4} {x4};
 \edge {theta} {x1,x2,x3,x4};

 %\edge {A}{z1,z2,z3,z4};
 %\edge {B}{x1,x2,x3,x4};
 %}
 \plate {plate1} {(x1)(x2)(x3)(x4)(z1)(z2)(z3)(z4)} {$N$}; %
 \end{tikzpicture}
 %}
\end{center} 

\begin{alignat*}{2}
    p(x, z; \ \theta)  &= \prod_{t=1}^T p(z_t \given z_{t-1} ; \ \mu_{z_{t-1}}) &&\times \prod_{t=1}^T p(x_t \given z_{t} ; \ \pi_{z_{t}}) \nonumber \\ 
    &= \prod_{t=1}^T \mu_{z_{t-1}, z_t} &&\times \prod_{t=1}^T \pi_{z_{t}, x_t}
\end{alignat*}

\end{frame}